\documentclass{article}

\usepackage[margin=1in]{geometry}

\usepackage{fontspec}
\setmainfont[Numbers=OldStyle]{Junicode}

\newfontface{\num}[Numbers=Uppercase]{Junicode}

\usepackage{nicefrac}

\newcounter{steps}

\newenvironment{recipe}[2]{% provide name and portions
	{\LARGE #1} \hspace{\fill} #2 portions
	\hrule
	\setcounter{steps}{1}
	\vspace{12pt}
}{}

\newenvironment{step}[0]{%
	\vspace{\parskip}\noindent\hspace{2em}
	\begin{minipage}{\linewidth}
		\begin{minipage}[t]{.1\linewidth}
			\arabic{steps}
		\end{minipage}
}{%
\addtocounter{steps}{1}\end{minipage}%
\vspace{6pt}%
}

\newenvironment{ingrs}[0]{%
	\begin{minipage}[t]{.3\linewidth}%
		\num%
		\setlength{\tabcolsep}{2pt}
		\begin{tabular}[t]{r l@{\hspace{2em}} l}
}{%
		\end{tabular}%
	\end{minipage}%
}

\newcommand{\ingr}[3]{#1 & #2 & #3 \\}
\newcommand{\itemingr}[2]{#1 & \multicolumn{2}{l}{#2}\\}
\newcommand{\tool}[1]{\multicolumn{3}{l}{#1}\\}

\newenvironment{stepdesc}[0]{%
	\begin{minipage}[t]{.5\linewidth}
}{%
	\end{minipage}
}

\begin{document}
  \begin{recipe}{Basic daal}{3}
    \begin{step}
      \begin{ingrs}
        \ingr{1}{cup}{dry rice}
        \ingr{1}{cup}{water}
        \tool{rice cooker}
      \end{ingrs}
      \begin{stepdesc}
        Prepare one cup of rice as normal. Turn on the rice cooker a bit before you start the rest of the process; the rice will take longer than the daal to cook.
      \end{stepdesc}
    \end{step}
    \begin{step}
      \begin{ingrs}
        \ingr{1}{cup}{red lentils}
        \tool{strainer}
      \end{ingrs}
      \begin{stepdesc}
        Rinse and strain one cup of lentils.
      \end{stepdesc}
    \end{step}
    \begin{step}
      \begin{ingrs}
        \ingr{3}{cups}{water}
        \ingr{1}{tsp}{turmeric}
        \ingr{1}{tsp}{salt}
        \ingr{1}{pinch}{chilli powder (optional)}
        \tool{four-quart pot}
      \end{ingrs}
      \begin{stepdesc}
        Put the lentils, water, salt and turmeric into a pot; mix well and bring to a boil.
      \end{stepdesc}
    \end{step}
    \begin{step}
      \begin{ingrs}
      \end{ingrs}
      \begin{stepdesc}
        When the water boils, turn down the heat to medium and let boil for 7 minutes.
      \end{stepdesc}
    \end{step}
    \begin{step}
      \begin{ingrs}
      \end{ingrs}
      \begin{stepdesc}
        After 7 minutes, remove the pot from the heat, place a lid on, and let it sit for another 5 minutes. In the meantime---
      \end{stepdesc}
    \end{step}
    \begin{step}
      \begin{ingrs}
        \ingr{1}{tsp}{ghee}
        \tool{small pan}
      \end{ingrs}
      \begin{stepdesc}
        Put a small amount of ghee in a small pan and melt it on the stove.
      \end{stepdesc}
    \end{step}
    \begin{step}
      \begin{ingrs}
        \ingr{1}{tbsp}{cumin seeds}
      \end{ingrs}
      \begin{stepdesc}
        When the pan is well hot, pour in the cumin seeds, stir quickly, and once the seeds start to pop (nearly immediately) remove the pan from the heat.
      \end{stepdesc}
    \end{step}
    \begin{step}
      \begin{ingrs}
      \end{ingrs}
      \begin{stepdesc}
        When the 5 minutes for the lentils are up, pour in the cumin seeds and ghee.
      \end{stepdesc}
    \end{step}
    \begin{step}
      \begin{ingrs}
        \itemingr{1}{lime}
      \end{ingrs}
      \begin{stepdesc}
        Cut the lime into quarters, and squeeze the juice into the daal mix. Mix well.
      \end{stepdesc}
    \end{step}
    \begin{step}
      \begin{ingrs}
      \end{ingrs}
      \begin{stepdesc}
        Serve on or alongside the rice.
      \end{stepdesc}
    \end{step}
  \end{recipe}
  \begin{recipe}{Basic sourdough bread}{10}
    \begin{step}
      \begin{ingrs}
        \ingr{65}{g}{flour}
        \ingr{65}{g}{water}
      \end{ingrs}
      \begin{stepdesc}
        Take your starter out of the fridge, remove about 125g and store in a jar for other things, and feed the starter with about 65g of flour and 65g of water.
      \end{stepdesc}
    \end{step}
    \begin{step}
      \begin{ingrs}
      \end{ingrs}
      \begin{stepdesc}
        Leave the starter on the counter to warm up and activate for 4--6 hours.
      \end{stepdesc}
    \end{step}
    \begin{step}
      \begin{ingrs}
        \ingr{125}{g}{starter}
        \ingr{330}{g}{flour}
        \ingr{220}{g}{water}
        \ingr{10}{g}{salt}
        \tool{mixing bowl}
      \end{ingrs}
      \begin{stepdesc}
        In a bowl, put together 125g of starter, 330g of flour, 220g of water, and 10g of salt.
      \end{stepdesc}
    \end{step}
    \begin{step}
      \begin{ingrs}
      \end{ingrs}
      \begin{stepdesc}
        Mix gently, switching to kneading at the end as necessary to mix. Don't knead any more than you have to.
      \end{stepdesc}
    \end{step}
    \begin{step}
      \begin{ingrs}
        \tool{cloth or wax wrap}
      \end{ingrs}
      \begin{stepdesc}
        Cover the bowl and let it sit on the counter for an hour.
      \end{stepdesc}
    \end{step}
    \begin{step}
      \begin{ingrs}
        \ingr{65}{g}{flour}
        \ingr{65}{g}{water}
      \end{ingrs}
      \begin{stepdesc}
        Re-feed your starter and put it back in the fridge.
      \end{stepdesc}
    \end{step}
    \begin{step}
      \begin{ingrs}
      \end{ingrs}
      \begin{stepdesc}
        After an hour, remove the cover, gently fold the dough over on itself a few times, turning a bit as you go. Put the cover back on and let it sit another hour.
      \end{stepdesc}
    \end{step}
    \begin{step}
      \begin{ingrs}
      \end{ingrs}
      \begin{stepdesc}
        Fold the dough again, then let it rest one more hour.
      \end{stepdesc}
    \end{step}
    \begin{step}
      \begin{ingrs}
      \end{ingrs}
      \begin{stepdesc}
        Fold the dough one last time, then put the dough in the refrigerator for 12--48 hours.
      \end{stepdesc}
    \end{step}
    \begin{step}
      \begin{ingrs}
        \tool{dutch oven}
        \tool{silicone liner or parchment paper}
      \end{ingrs}
      \begin{stepdesc}
        When you have about four hours to rise and bake, take the dough out of the fridge, punch it down, pick it up and shape it in your hands into a ball, and set it down in a lined dutch oven. Let it rest and warm up on the counter with the lid on for three hours.
      \end{stepdesc}
    \end{step}
    \begin{step}
      \begin{ingrs}
      \end{ingrs}
      \begin{stepdesc}
        After three hours, preheat your oven to 500 degrees F.
      \end{stepdesc}
    \end{step}
    \begin{step}
      \begin{ingrs}
      \end{ingrs}
      \begin{stepdesc}
        When the oven is hot enough, score the top of your dough in an X with a knife, then put the lid back on and put it in the oven. Immediately reduce the oven temperature to 450 degrees, and set a timer for 45 minutes.
      \end{stepdesc}
    \end{step}
    \begin{step}
      \begin{ingrs}
      \end{ingrs}
      \begin{stepdesc}
        After 45 minutes, open the oven, remove the lid on the dutch oven, and close the oven back up again. Let it bake another 15 minutes.
      \end{stepdesc}
    \end{step}
    \begin{step}
      \begin{ingrs}
      \end{ingrs}
      \begin{stepdesc}
        After 15 minutes, the bread is done! Take it out of the oven and the dutch oven, and put it on a cooling rack to cool completely before serving or storing it.
      \end{stepdesc}
    \end{step}
  \end{recipe}
\end{document}