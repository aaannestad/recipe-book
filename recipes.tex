\documentclass{article}

\usepackage[margin=1in]{geometry}

\usepackage{fontspec}
\setmainfont[Numbers=OldStyle]{Junicode}

\newcounter{steps}

\newenvironment{recipe}[2]{% provide name and portions
	{\LARGE #1} \hskip #2 portions\\
	\hline
	\setcounter{steps}{1}
}{enddef}

\begin{document}
  \begin{recipe}[\url{today.com/recipes/most-basic-dal-recipe-t154897}]{Basic daal}{3}
    \begin{step}
      \begin{ingrs}
        \ingr{1}{cup}{dry rice}
        \ingr{1}{cup}{water}
        \tool{rice cooker}
      \end{ingrs}
      \begin{stepdesc}
        Prepare one cup of rice as normal. Turn on the rice cooker a bit before you start the rest of the process; the rice will take longer than the daal to cook.
      \end{stepdesc}
    \end{step}
    \begin{step}
      \begin{ingrs}
        \ingr{1}{cup}{red lentils}
        \tool{strainer}
      \end{ingrs}
      \begin{stepdesc}
        Rinse and strain one cup of lentils.
      \end{stepdesc}
    \end{step}
    \begin{step}
      \begin{ingrs}
        \ingr{3}{cups}{water}
        \ingr{1}{tsp}{turmeric}
        \ingr{1}{tsp}{salt}
        \ingr{1}{pinch}{chilli powder (optional)}
        \tool{four-quart pot}
      \end{ingrs}
      \begin{stepdesc}
        Put the lentils, water, salt and turmeric into a pot; mix well and bring to a boil.
      \end{stepdesc}
    \end{step}
    \begin{step}
      \begin{ingrs}
      \end{ingrs}
      \begin{stepdesc}
        When the water boils, turn down the heat to medium and let boil for 7 minutes.
      \end{stepdesc}
    \end{step}
    \begin{step}
      \begin{ingrs}
      \end{ingrs}
      \begin{stepdesc}
        After 7 minutes, remove the pot from the heat, place a lid on, and let it sit for another 5 minutes. In the meantime---
      \end{stepdesc}
    \end{step}
    \begin{step}
      \begin{ingrs}
        \ingr{1}{tbsp}{ghee}
        \tool{small pan}
      \end{ingrs}
      \begin{stepdesc}
        Put a small amount of ghee in a small pan and melt it on the stove.
      \end{stepdesc}
    \end{step}
    \begin{step}
      \begin{ingrs}
        \ingr{1}{tbsp}{cumin seeds}
      \end{ingrs}
      \begin{stepdesc}
        When the pan is well hot, pour in the cumin seeds, stir quickly, and once the seeds start to pop (nearly immediately) remove the pan from the heat.
      \end{stepdesc}
    \end{step}
    \begin{step}
      \begin{ingrs}
      \end{ingrs}
      \begin{stepdesc}
        When the 5 minutes for the lentils are up, pour in the cumin seeds and ghee.
      \end{stepdesc}
    \end{step}
    \begin{step}
      \begin{ingrs}
        \itemingr{1}{lime}
      \end{ingrs}
      \begin{stepdesc}
        Cut the lime into quarters, and squeeze the juice into the daal mix. Mix well.
      \end{stepdesc}
    \end{step}
    \begin{step}
      \begin{ingrs}
      \end{ingrs}
      \begin{stepdesc}
        Serve on or alongside the rice.
      \end{stepdesc}
    \end{step}
  \end{recipe}
  \begin{recipe}[\url{kingarthurbaking.com/recipes/no-knead-sourdough-bread-recipe}]{Basic sourdough bread}{10}
    \begin{step}
      \begin{ingrs}
        \ingr{65}{g}{flour}
        \ingr{65}{g}{water}
      \end{ingrs}
      \begin{stepdesc}
        Take your starter out of the fridge, remove about 125g and store in a jar for other things, and feed the starter with about 65g of flour and 65g of water.
      \end{stepdesc}
    \end{step}
    \begin{step}
      \begin{ingrs}
      \end{ingrs}
      \begin{stepdesc}
        Leave the starter on the counter to warm up and activate for 4--6 hours.
      \end{stepdesc}
    \end{step}
    \begin{step}
      \begin{ingrs}
        \ingr{125}{g}{starter}
        \ingr{330}{g}{flour}
        \ingr{220}{g}{water}
        \ingr{10}{g}{salt}
        \tool{mixing bowl}
      \end{ingrs}
      \begin{stepdesc}
        In a bowl, put together 125g of starter, 330g of flour, 220g of water, and 10g of salt.
      \end{stepdesc}
    \end{step}
    \begin{step}
      \begin{ingrs}
      \end{ingrs}
      \begin{stepdesc}
        Mix gently, switching to kneading at the end as necessary to mix. Don't knead any more than you have to.
      \end{stepdesc}
    \end{step}
    \begin{step}
      \begin{ingrs}
        \tool{cloth or wax wrap}
      \end{ingrs}
      \begin{stepdesc}
        Cover the bowl and let it sit on the counter for an hour.
      \end{stepdesc}
    \end{step}
    \begin{step}
      \begin{ingrs}
        \ingr{65}{g}{flour}
        \ingr{65}{g}{water}
      \end{ingrs}
      \begin{stepdesc}
        Re-feed your starter and put it back in the fridge.
      \end{stepdesc}
    \end{step}
    \begin{step}
      \begin{ingrs}
      \end{ingrs}
      \begin{stepdesc}
        After an hour, remove the cover, gently fold the dough over on itself a few times, turning a bit as you go. Put the cover back on and let it sit another hour.
      \end{stepdesc}
    \end{step}
    \begin{step}
      \begin{ingrs}
      \end{ingrs}
      \begin{stepdesc}
        Fold the dough again, then let it rest one more hour.
      \end{stepdesc}
    \end{step}
    \begin{step}
      \begin{ingrs}
      \end{ingrs}
      \begin{stepdesc}
        Fold the dough one last time, then put the dough in the refrigerator for 12--48 hours.
      \end{stepdesc}
    \end{step}
    \begin{step}
      \begin{ingrs}
        \tool{dutch oven}
        \tool{silicone liner or parchment paper}
      \end{ingrs}
      \begin{stepdesc}
        When you have about four hours to rise and bake, take the dough out of the fridge, punch it down, pick it up and shape it in your hands into a ball, and set it down in a lined dutch oven. Let it rest and warm up on the counter with the lid on for three hours.
      \end{stepdesc}
    \end{step}
    \begin{step}
      \begin{ingrs}
      \end{ingrs}
      \begin{stepdesc}
        After three hours, preheat your oven to 500 degrees F.
      \end{stepdesc}
    \end{step}
    \begin{step}
      \begin{ingrs}
      \end{ingrs}
      \begin{stepdesc}
        When the oven is hot enough, score the top of your dough in an X with a knife, then put the lid back on and put it in the oven. Immediately reduce the oven temperature to 450 degrees, and set a timer for 45 minutes.
      \end{stepdesc}
    \end{step}
    \begin{step}
      \begin{ingrs}
      \end{ingrs}
      \begin{stepdesc}
        After 45 minutes, open the oven, remove the lid on the dutch oven, and close the oven back up again. Let it bake another 15 minutes.
      \end{stepdesc}
    \end{step}
    \begin{step}
      \begin{ingrs}
      \end{ingrs}
      \begin{stepdesc}
        After 15 minutes, the bread is done! Take it out of the oven and the dutch oven, and put it on a cooling rack to cool completely before serving or storing it.
      \end{stepdesc}
    \end{step}
  \end{recipe}
  \begin{recipe}[\url{https://growagoodlife.com/apple-scrap-vinegar/}]{Apple cider vinegar}{many}
    \begin{step}
      \begin{ingrs}
        \ingr{12}{apples worth}{apple scraps}
        \tool{quart jar}
      \end{ingrs}
      \begin{stepdesc}
        Put about 12 apples' worth of cores and other scraps in a quart jar; squishing them down as much as you can.
      \end{stepdesc}
    \end{step}
    \begin{step}
      \begin{ingrs}
        \ingr{1}{cup}{water}
        \ingr{2}{tbsp}{sugar}
        \ingr{1}{tbsp}{live vinegar (optional)}
      \end{ingrs}
      \begin{stepdesc}
        Mix about a cup of water with 2 tbsp of sugar, then pour over the apple scraps. Add a tablespoon of vinegar from a previous batch or another live vinegar source if you want, for a bit faster maturation and somewhat lower chance of failure. Add water to fill up the rest of the jar, leaving a few inches of space at the top. You may need a fermentation weight or similar tool to keep the scraps all submerged.
      \end{stepdesc}
    \end{step}
    \begin{step}
      \begin{ingrs}
        \tool{coffee filter or paper towel}
        \tool{rubber band or string or mason jar band}
      \end{ingrs}
      \begin{stepdesc}
        Cover the jar with a breathable cover, and leave it to sit in a warm place out of sunlight. Wait a few days for it to start to visibly bubble.
      \end{stepdesc}
    \end{step}
    \begin{step}
      \begin{ingrs}
      \end{ingrs}
      \begin{stepdesc}
        Once it's visibly bubbling, stir it once a day to oxygenate the water until it stops bubbling---about two weeks or so.
      \end{stepdesc}
    \end{step}
    \begin{step}
      \begin{ingrs}
        \tool{mixing bowl or other receptacle}
        \tool{cheesecloth or strainer bag}
      \end{ingrs}
      \begin{stepdesc}
        Once the vinegar stops bubbling, strain the scraps out through a cloth into a bowl. Squeeze the scraps to liberate as much remaining liquid as you can. Discard the scraps.
      \end{stepdesc}
    \end{step}
    \begin{step}
      \begin{ingrs}
      \end{ingrs}
      \begin{stepdesc}
        Pour the vinegar back into the jar, cover with a breathable cover, and let it ferment without stirring for another two to four weeks, until it has the sourness you're looking for.
      \end{stepdesc}
    \end{step}
    \begin{step}
      \begin{ingrs}
      \end{ingrs}
      \begin{stepdesc}
        Once it tastes like you want it to, close it with a regular lid or put it into a more pourable bottle, and leave it in the fridge for up to a year.
      \end{stepdesc}
    \end{step}
  \end{recipe}
  \begin{recipe}[\url{mychefsapron.com/one-hearty-vegetable-stew-recipe-vegan-gf/}]{Potato and squash stew}{4}
    \begin{step}
      \begin{ingrs}
        \ingr{1}{tbsp}{oil}
        \tool{four-quart pot or dutch oven}
      \end{ingrs}
      \begin{stepdesc}
        Put a tablespoon of oil in a pot on the stove. Don't turn the stove on yet.
      \end{stepdesc}
    \end{step}
    \begin{step}
      \begin{ingrs}
        \itemingr{1}{honeynut squash}
        \itemingr{3}{potatoes}
        \itemingr{1}{onion}
      \end{ingrs}
      \begin{stepdesc}
        Cut up three potatoes, a honeynut squash (or half a butternut squash), and an onion; put the chopped pieces in the pot as you go. The squash is easier to peel if you microwave it for 30 seconds beforehand. (There's no need to peel the potatoes.)
      \end{stepdesc}
    \end{step}
    \begin{step}
      \begin{ingrs}
        \ingr{1}{tsp}{salt}
      \end{ingrs}
      \begin{stepdesc}
        Put a teaspoon of salt in with the vegetables, turn on the heat, and stir and cook for five minutes.
      \end{stepdesc}
    \end{step}
    \begin{step}
      \begin{ingrs}
        \ingr{2}{cloves}{garlic}
      \end{ingrs}
      \begin{stepdesc}
        Near the end of the five minutes, chop up two cloves of garlic.
      \end{stepdesc}
    \end{step}
    \begin{step}
      \begin{ingrs}
      \end{ingrs}
      \begin{stepdesc}
        After five minutes, put the lid on and leave it to cook for another five minutes.
      \end{stepdesc}
    \end{step}
    \begin{step}
      \begin{ingrs}
        \ingr{2}{cups}{peanut milk}
        \ingr{1}{tsp}{turmeric}
        \ingr{1}{tsp}{ground pepper}
      \end{ingrs}
      \begin{stepdesc}
        Open the pot, pour in the peanut milk, and stir in the garlic, turmeric, and pepper. Let it come to a simmer.
      \end{stepdesc}
    \end{step}
    \begin{step}
      \begin{ingrs}
      \end{ingrs}
      \begin{stepdesc}
        Once the stew is simmering, cover the pan and cook for ten minutes, or longer if the potatoes aren't yet soft.
      \end{stepdesc}
    \end{step}
    \begin{step}
      \begin{ingrs}
        \ingr{3}{tbsp}{nutritional yeast}
      \end{ingrs}
      \begin{stepdesc}
        Once the potatoes are done through, mix in the nutritional yeast and let it cook for two further minutes.
      \end{stepdesc}
    \end{step}
    \begin{step}
      \begin{ingrs}
      \end{ingrs}
      \begin{stepdesc}
        After two minutes, remove it from the heat and serve.
      \end{stepdesc}
    \end{step}
  \end{recipe}
\end{document}